\section{Introduction}

Presentation technology has a significant impact on how we learn and teach. Today, there are largely two modes of presentation technology used in classrooms: inking on surface and projecting prepared slides. \\
35mm slide projectors first came into widespread use during the 1950s, but recently the popularization of slide authoring tools like PowerPoint, Keynote or Google Slides made electronic slides became much more common and easy. Despite its advantages--e.g., easy to share, archive, include multimedia--slides also have critical drawbacks. 
(1) Slide presentations are \textbf{rigid}: All of the editing and preparation is done ahead of time and fixed at the time of the presentation. There is no flexibility to change the order or content of the presentation during performance. 
(2) Slide presentations are \textbf{discrete}: Information is divided and presented in chunks. First the entire content is divided into separate slides, and within each slide text or graphics are presented in chunks, usually using animation effects. The appearance or transition of information is sudden and discrete.
(3) Slide presentations are \textbf{indirect}: The action of the presenter (i.e., pressing "next" to advance the slide) is removed from the effect on the content. For example, this makes it easy for presenters to forget or skip an animation sequence.\\

Inking on surface is an alternative or supplement. Inking includes blackboard or transparency and overhead projector. Inking complements the characteristics of slides. Inking is flexible, continuous and direct, but this has its own downsides.
(1) Inking is \textbf{flexible} since all of the presentation is done on the fly. But, this means there is a lot of cognitive overload for the presenter in order to decide the content, layout and order of the presentation on the fly. 
(2) Inking is \textbf{continuous}: Since the presenter writes or draws in real time, content is presented in a continuous way. This is useful for information-loaded contents or where order within the content is important, for example, derivation of a math formula or describing a temporal process. However, it is difficult to time the presentation or present a lot of information. Inking lectures tend to be slower paced [citation].
Finally, (3) inking is direct. Users actions (drawing, erasing, underlying) are all directly translated into content. This requires a lot of attention, skill on the part of the presenter. Often, presenters are reluctant to write because of their messy handwriting. Also content is limited to text or simple diagrams.

There are previous work to blend the two modes. [Classroom Presenter] or recent versions of [PowerPoint] allow you to ink on slides. But these tools treat the two modes as separate. Ink and slide remain separate layers retaining their characteristics. The slide layer remains rigid and fixed; and ink is placed on top of it.

\textbf{We propose inking as a main modality to present slide contents in a continuous, flexible and direct manner}. By inking over to reveal content, users regain flexibility, continuity and direct control over the presentation style, without losing the elegance of the prepared content.  

