\section{Introduction}

Content presentation is an integral part of instruction. So presentation technology has an important role in how we learn and teach. Today, there are largely two modes of presentation technology used in classrooms: blackboards and projected electronic slides. 

Slide presentations are \textbf{rigid}: all of the editing and preparation is done ahead of time and fixed at the time of the presentation. Fixed content and order; there is no flexibility.
Slide presentations are \textbf{discrete}: information is presented chunks at a time. 
Slide presentations are \textbf{indirect}: 

The alternative is inking, includes blackboard or transparency and overhead projector. 
Inking is flexible, continuous and direct; large overhead; messy; little control over the elements that are drawn.

There are some tools \cite{ClassroomPresenter} that attempted to bring the two modalities. But, they put one on top of each other. The slide component remained rigid and fixed; and on top of it the ink remained.

Inking as a main modality to present slides. This brings back flexibility/ continuity and direct control; but maintain the "elegance" of the prepared content.  

